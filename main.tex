\documentclass[a4paper,12pt]{article}

\usepackage{phypreamble}


\newcommand{\phyitmogroup}{M3204}
\newcommand{\phyitmostudent}{Иванов И. И.}
\newcommand{\phyitmoteacher}{Петров П. П.}
\newcommand{\phyitmonumber}{1}
\newcommand{\phyitmotitle}{Влияние радиоволн на интеллект человека}

\begin{document}
\import{./}{titlepage}

% \section{Цель работы}

% \section{Задачи, решаемые при выполнении работы}

% \section{Объект исследования}

% \section{Метод экспериментального исследования}

% \section{Рабочие формулы и исходные данные}

% \section{Измерительные приборы}

% \noindent\begin{tabularx}{\textwidth}{ | >{\rownum}r | X | X | X | X | }
%   \toprule
%   & Наименование & Тип прибора & Используемый диапазон & Погрешность прибора \\ \midrule
%   & & & & \\ \midrule
%   & & & & \\
%   \bottomrule
% \end{tabularx}

% \section{Схема установки}

% \section{Результаты прямых измерений и их обработки}

% \section{Расчёт результатов косвенных измерений}

% \section{Расчёт погрешностей измерений}

% \section{Графики}

% \section{Окончательные результаты}

% \section{Выводы и анализ результатов работы}

% \section{Дополнительные задания}

% \section{Выполнение дополнительных заданий}

% \section{Замечания преподавателя}


\end{document}